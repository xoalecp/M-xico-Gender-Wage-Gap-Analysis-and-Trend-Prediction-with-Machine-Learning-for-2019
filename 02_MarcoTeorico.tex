\newpage
\section{Marco Te\'orico}

\subsection{Estudios Actuales}
Las brechas salariales de g\'enero en M\'exico se han estudiado asiduamente. Entre los primeros an\'alisis de las brechas salariales se encuentra el de (Alarc\'on, 1994), quienes utilizaron la muestra urbana de la Encuesta Nacional de Ingreso y Gasto de los Hogares (ENIGH) de 1984, 1989 y 1992. En sus trabajos encontraron que en 1984 las mujeres ganaban 23.3\% menos que los hombres; hacia 1989 esta cifra hab\'ia aumentado a 28.4\%, y en 1992 disminuy\'o a 25.3\%. Siguiendo la l\'inea de investigaci\'on de Oaxaca (1973) y Blinder (1973), estos autores realizaron una descomposici\'on de la brecha salarial en la media, mediante la estimaci\'on de ecuaciones de Mincer (1974) , para analizar tanto la parte de la brecha originada por caracter\'isticas observables como la parte provocada por los retornos a tales caracter\'isticas. Encontraron tambi\'en que s\'olo 27.5\% de la brecha se explicaba por diferencias en capital humano en 1984, mientras que en 1989 la proporci\'on fue de 14.4\% y en 1992 de 21.2\%; es decir que entre 70 y 85\% de las brechas se deb\'ian a diferencias en los retornos al capital humano, lo cual podr\'ia sugerir discriminaci\'on en contra de las mujeres o diferencias en productividad que no fueron controladas en la regresi\'on.

Por su parte, Brown, Pagan y Rodr\'iguez-Oreggia (1999) analizaron los cambios en las brechas salariales entre 1987 y 1993 con base en datos de los terceros trimestres de la Encuesta Nacional de Empleo Urbano. Ellos realizaron una descomposici\'on de Wellington (1993) de los cambios de la brecha en el tiempo, y una descomposici\'on de Oaxaca-Blinder para analizar el efecto de la estructura ocupacional en la brecha. Encuentran que la brecha creci\'o en el periodo de un nivel inicial de 20.8\%, en 1987, a 22\%, en 1993. Este crecimiento en la brecha se debi\'o a cambios en las dotaciones, pues a causa de los cambios en los retornos la brecha se hubiese cerrado. Los autores tambi\'en encontraron que la mayor parte de la brecha se gener\'o por diferencias en retornos. Sin embargo, lo interesante de sus hallazgos es que la inclusi\'on de controles ocupacionales aumenta la proporci\'on de la brecha explicada por diferencias a los retornos, lo cual, seg\'un explican, puede ser resultado de la poca desagregaci\'on de las categor\'ias ocupacionales. Es decir, la segregaci\'on ocupacional disminuye la brecha salarial en M\'exico, lo cual contrasta con los resultados de otros pa\'ises (Blau, Simpson y Anderson, 1998).

M\'as recientemente, Pagan y Ullibarri (2000) analizaron la desigualdad salarial entre hombres y mujeres por medio del \'indice de Jenkins, corrigiendo por selecci\'on en la participaci\'on laboral de las mujeres. Con base en datos de la ENEU del tercer trimestre de 1995, encontraron que existe mayor desigualdad entre personas con alta y baja escolaridad, as\'i como entre aquellas con mayor experiencia. Por su parte, elaboraron una descomposici\'on del tipo Oaxaca-Blinder mediante la ENIGH 2000, corrigiendo por sesgo de selecci\'on con la metodolog\'ia de Heckman (1974, 1979). Los autores fueron los primeros en incluir en su an\'alisis zonas urbanas y rurales. Hallaron que 85\% de la brecha se debe a diferencias en retornos y que \'esta es mayor en zonas rurales; de hecho, el efecto de las dotaciones otorga una ventaja a las mujeres. Por \'ultimo, Garc\'ia y Mendoza (2009) elaboraron una descomposici\'on de Oaxaca-Blinder sin corregir por sesgo de selecci\'on y usando datos de la ENOE 2006. Su hallazgo fue una brecha salarial de 12.4\% y, al contrario que el resto de la bibliograf\'ia, determinaron que 87.6\% de la brecha se explica por diferencias en las dotaciones, seg\'un la cual el 12.4\% restante corresponde a diferencias en los retornos a \'estas.

\subsection{An\'alisis desde el Contexto Social}
Para el caso de M\'exico, la \'unica causa explorada de la brecha no explicada o la brecha de retornos ha sido la liberalizaci\'on comercial Artecona y Cunningham (2002) encuentran evidencia que sugiere que la liberalizaci\'on comercial provoc\'o una disminuci\'on de la discriminaci\'on en las empresas manufactureras que fueron más afectadas por la liberalizaci\'on. Por su parte, Aguayo-T\'ellez, Airola y Juhn (2010) encuentran que la liberalizaci\'on comercial no afect\'o los salarios, pero s\'i tuvo un efecto en el empleo de las mujeres. De esta manera, la evidencia sobre esta posible causal no es muy concluyente. Consideramos que investigaciones futuras deben abordar la cuesti\'on de las causas de los cambios en las brechas salariales de g\'enero y de la existencia de "pisos pegajosos" y "techos de cristal". Creemos que el mecanismo expuesto por De la Rica et al. (2008) podr\'ia tambi\'en estar operando en el caso mexicano. Por otra parte, y siguiendo a Arulampalam et al. (2007) y Christofides et al. (2013) tambi\'en es necesario explorar el efecto que tuvieron los cambios institucionales de las d\'ecadas de 1980 y 1990 (como la ca\'ida del salario m\'inimo real, la negociaci\'on colectiva de los salarios y la cobertura sindical) en la brecha salarial de g\'enero. Por ejemplo, Arulampalam et al. (2007) sugieren que la dispersi\'on salarial está negativamente relacionada con los "techos de cristal" y positivamente relacionada con los "pisos pegajosos". Si este resultado fuese generalizable a M\'exico, deber\'iamos observar que la disminuci\'on de la desigualdad entre 2000 y 2010 se hubiera reflejado en mayores "techos de cristal" y menores "pisos pegajosos", lo cual contrasta con nuestros resultados. As\'i, es importante analizar c\'omo la reducci\'on observada en la desigualdad salarial en la d\'ecada pasada afect\'o la brecha salarial de g\'enero en el contexto mexicano. Otra posible l\'inea de investigaci\'on se abre en torno al hallazgo sistemático en la bibliograf\'ia sobre M\'exico de que la segregaci\'on ocupacional de hecho favorece la brecha salarial de g\'enero, lo cual es congruente con los resultados de Australia (Bar\'on y Cobb-Clark, 2010), pero no con los de otros pa\'ises (Blau, Simpson y Anderson, 1998), as\'i como con la creencia generalizada de que la segregaci\'on ocupacional es una causal de la existencia de la brecha salarial. Un mayor entendimiento de estas causales nos dar\'ia mejores fundamentos para diseñar pol\'iticas p\'ublicas que promuevan la igualdad de g\'enero en el mercado laboral\'
