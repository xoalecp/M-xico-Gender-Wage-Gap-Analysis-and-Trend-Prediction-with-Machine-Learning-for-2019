\newpage
\section{Planteamiento del Problema}

\subsection{Descriptci\'on del problema}
En el an\'alisis de mercado laboral es posible observar la diversificaci\'on de ocupaciones basadas en la delimitaci\'on geogr\'afica y, en consecuencia, diferentes remuneraciones se generan de acuerdo con el tipo de trabajo. Estas remuneraciones a su vez poseen ciertas diferencias dentro de un mercado competitivo; las mismas se sustentan en la oferta y la demanda de una determinada profesi\'on, e incluso con el nivel de productividad que el capital humano posee. Dada la premisa anterior, cualquier factor que no impacte directamente en el nivel de productividad de un individuo, no deber\'ia afectar a su vez en la remuneraci\'on que el mismo perciba; dicho de otra manera, la religi\'on, color de piel o GENERO no deber\'ian generar diferencias en las remuneraciones. (Rodr\'iguez-P\'erez & Castro-Lugo, 2014)
A pesar de que en las \'ultimas d\'ecadas las mujeres han aumentado su nivel de educaci\'on y ocupado posiciones laborales de la misma \'indole que los hombres, diferentes organismos han demostrado que existe una gran brecha salaria de g\'enero en el Mundo (Brown, Pag\'an, & Rodr\'iguez, 1999). Los numero son tan alarmantes que uno de los objetivos del G-20, es reducir la brecha de genero en un 25\% para el 2025. (Website of the Ministry of Foreign Affairs of Japan, 2019)

\subsection{Justificaci\'on}
En Am\'erica Latina, M\'exico se encuentra en el \'ultimo lugar en materia de igualdad de g\'enero. de acuerdo con el \'indice de brechas de g\'enero globales (L\'opez-Carlos & Zahidi, 2005), “entre los 56 pa\'ises estudiados M\'exico se encuentra en el lugar n\'umero 52, s\'olo por encima de India, Corea, Jordania, Pakist\'an, Turqu\'ia y Egipto”.


\subsection{Objetivos}
\subsubsection{Objetivo General}
\begin{itemize}
    \item Realizar un an\'alisis predictivo del salario de mujeres y hombres en el año 2019
\end{itemize}
\subsubsection{Objetivos Particulares}
\begin{itemize}
    \item Generar un dataset de entrenamiento a partir de bases de datos abiertas del gobierno
    \item Aplicar diferentes algoritmos de regresi\'on
    \item Generar un modelo con el menor error posible en la predicci\'on del salario
\end{itemize}
